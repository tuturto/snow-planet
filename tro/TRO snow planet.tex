\documentclass{tufte-book}

\hypersetup{colorlinks}% uncomment this line if you prefer colored hyperlinks (e.g., for onscreen viewing)

%%
% Book metadata
\title{TRO: Kiiminki}
\author{Tuukka Turto}
\publisher{N/A}

%%
% If they're installed, use Bergamo and Chantilly from www.fontsite.com.
% They're clones of Bembo and Gill Sans, respectively.
%\IfFileExists{bergamo.sty}{\usepackage[osf]{bergamo}}{}% Bembo
%\IfFileExists{chantill.sty}{\usepackage{chantill}}{}% Gill Sans

%\usepackage{microtype}

%%
% Just some sample text
\usepackage{gensymb}
\usepackage{lipsum}
\usepackage{geometry}

\geometry{paper=a4paper}

%%
% For nicely typeset tabular material
\usepackage{booktabs}

%%
% For graphics / images
\usepackage{graphicx}
\setkeys{Gin}{width=\linewidth,totalheight=\textheight,keepaspectratio}
\graphicspath{{graphics/}}

% The fancyvrb package lets us customize the formatting of verbatim
% environments.  We use a slightly smaller font.
\usepackage{fancyvrb}
\fvset{fontsize=\normalsize}

%%
% Prints argument within hanging parentheses (i.e., parentheses that take
% up no horizontal space).  Useful in tabular environments.
\newcommand{\hangp}[1]{\makebox[0pt][r]{(}#1\makebox[0pt][l]{)}}

%%
% Prints an asterisk that takes up no horizontal space.
% Useful in tabular environments.
\newcommand{\hangstar}{\makebox[0pt][l]{*}}

%%
% Prints a trailing space in a smart way.
\usepackage{xspace}

% Prints the month name (e.g., January) and the year (e.g., 2008)
\newcommand{\monthyear}{%
  \ifcase\month\or January\or February\or March\or April\or May\or June\or
  July\or August\or September\or October\or November\or
  December\fi\space\number\year
}


% Prints an epigraph and speaker in sans serif, all-caps type.
\newcommand{\openepigraph}[2]{%
  %\sffamily\fontsize{14}{16}\selectfont
  \begin{fullwidth}
  \sffamily\large
  \begin{doublespace}
  \noindent\allcaps{#1}\\% epigraph
  \noindent\allcaps{#2}% author
  \end{doublespace}
  \end{fullwidth}
}

% Inserts a blank page
\newcommand{\blankpage}{\newpage\hbox{}\thispagestyle{empty}\newpage}

\usepackage{units}

% Typesets the font size, leading, and measure in the form of 10/12x26 pc.
\newcommand{\measure}[3]{#1/#2$\times$\unit[#3]{pc}}

% Macros for typesetting the documentation
\newcommand{\hlred}[1]{\textcolor{Maroon}{#1}}% prints in red
\newcommand{\hangleft}[1]{\makebox[0pt][r]{#1}}
\newcommand{\hairsp}{\hspace{1pt}}% hair space
\newcommand{\hquad}{\hskip0.5em\relax}% half quad space
\newcommand{\TODO}{\textcolor{red}{\bf TODO!}\xspace}
\newcommand{\ie}{\textit{i.\hairsp{}e.}\xspace}
\newcommand{\eg}{\textit{e.\hairsp{}g.}\xspace}
\newcommand{\na}{\quad--}% used in tables for N/A cells
\providecommand{\XeLaTeX}{X\lower.5ex\hbox{\kern-0.15em\reflectbox{E}}\kern-0.1em\LaTeX}
\newcommand{\tXeLaTeX}{\XeLaTeX\index{XeLaTeX@\protect\XeLaTeX}}
% \index{\texttt{\textbackslash xyz}@\hangleft{\texttt{\textbackslash}}\texttt{xyz}}
\newcommand{\tuftebs}{\symbol{'134}}% a backslash in tt type in OT1/T1
\newcommand{\doccmdnoindex}[2][]{\texttt{\tuftebs#2}}% command name -- adds backslash automatically (and doesn't add cmd to the index)
\newcommand{\doccmddef}[2][]{%
  \hlred{\texttt{\tuftebs#2}}\label{cmd:#2}%
  \ifthenelse{\isempty{#1}}%
    {% add the command to the index
      \index{#2 command@\protect\hangleft{\texttt{\tuftebs}}\texttt{#2}}% command name
    }%
    {% add the command and package to the index
      \index{#2 command@\protect\hangleft{\texttt{\tuftebs}}\texttt{#2} (\texttt{#1} package)}% command name
      \index{#1 package@\texttt{#1} package}\index{packages!#1@\texttt{#1}}% package name
    }%
}% command name -- adds backslash automatically
\newcommand{\doccmd}[2][]{%
  \texttt{\tuftebs#2}%
  \ifthenelse{\isempty{#1}}%
    {% add the command to the index
      \index{#2 command@\protect\hangleft{\texttt{\tuftebs}}\texttt{#2}}% command name
    }%
    {% add the command and package to the index
      \index{#2 command@\protect\hangleft{\texttt{\tuftebs}}\texttt{#2} (\texttt{#1} package)}% command name
      \index{#1 package@\texttt{#1} package}\index{packages!#1@\texttt{#1}}% package name
    }%
}% command name -- adds backslash automatically
\newcommand{\docopt}[1]{\ensuremath{\langle}\textrm{\textit{#1}}\ensuremath{\rangle}}% optional command argument
\newcommand{\docarg}[1]{\textrm{\textit{#1}}}% (required) command argument
\newenvironment{docspec}{\begin{quotation}\ttfamily\parskip0pt\parindent0pt\ignorespaces}{\end{quotation}}% command specification environment
\newcommand{\docenv}[1]{\texttt{#1}\index{#1 environment@\texttt{#1} environment}\index{environments!#1@\texttt{#1}}}% environment name
\newcommand{\docenvdef}[1]{\hlred{\texttt{#1}}\label{env:#1}\index{#1 environment@\texttt{#1} environment}\index{environments!#1@\texttt{#1}}}% environment name
\newcommand{\docpkg}[1]{\texttt{#1}\index{#1 package@\texttt{#1} package}\index{packages!#1@\texttt{#1}}}% package name
\newcommand{\doccls}[1]{\texttt{#1}}% document class name
\newcommand{\docclsopt}[1]{\texttt{#1}\index{#1 class option@\texttt{#1} class option}\index{class options!#1@\texttt{#1}}}% document class option name
\newcommand{\docclsoptdef}[1]{\hlred{\texttt{#1}}\label{clsopt:#1}\index{#1 class option@\texttt{#1} class option}\index{class options!#1@\texttt{#1}}}% document class option name defined
\newcommand{\docmsg}[2]{\bigskip\begin{fullwidth}\noindent\ttfamily#1\end{fullwidth}\medskip\par\noindent#2}
\newcommand{\docfilehook}[2]{\texttt{#1}\index{file hooks!#2}\index{#1@\texttt{#1}}}
\newcommand{\doccounter}[1]{\texttt{#1}\index{#1 counter@\texttt{#1} counter}}

% Generates the index
\usepackage{makeidx}
\makeindex

\begin{document}

% Front matter
\frontmatter

% r.3 full title page
\maketitle


% v.4 copyright page
\newpage
\begin{fullwidth}
~\vfill
\thispagestyle{empty}
\setlength{\parindent}{0pt}
\setlength{\parskip}{\baselineskip}
Copyright \copyright\ \the\year\ \thanklessauthor

\par\smallcaps{Published by \thanklesspublisher}

\par\smallcaps{tufte-latex.googlecode.com}

\par Licensed under CC BY-SA 4.0.\index{license}

\par\textit{First printing, \monthyear}
\end{fullwidth}

% r.5 contents
\setcounter{tocdepth}{2}
\tableofcontents

%%
% Start the main matter (normal chapters)
\mainmatter


\chapter{Kiiminki System}
\label{ch:kiiminki-system}

\newthought{Kiiminki} is a star system located at xxx-yyy.

\section{Kiiminki Star}

\newthought{Kiiminki} is a star.

\bigskip
\begin{minipage}{\textwidth}
\begin{center}
\begin{tabular}{llll}
\toprule
System Name: & \multicolumn{3}{l}{Kiiminki} \\
Star Name: & \multicolumn{3}{l}{Kiiminki} \\
Star Type: & \multicolumn{3}{l}{K0V} \\
\quad Recharge Time: & \multicolumn{3}{l}{191 hours} \\
\quad Proximity Limit: & \multicolumn{3}{l}{549 564 113 km} \\
\quad Transit Time: & \multicolumn{3}{l}{5.48 days} \\

\multicolumn{4}{l}{Planets} \\
\quad 0.32 AU & J\"{a}nni      & 4.16 AU  & Keyritty \\
\quad 0.56 AU & Kakkuri    & 8.0 AU   & Meiko \\
\quad 0.8 AU  & Kuolimo    & 15.68 AU & Maintainen \\
\quad 1.28 AU & Jalanti    & 31.04 AU & Kuorinka \\
\quad 2.24 AU & Iso-Lankko & & \\

\bottomrule
\end{tabular}
\end{center}
\end{minipage}

\section{J\"{a}nni}

\newthought{J\"{a}nni} is the first planet of Kiiminki system.

\bigskip
\begin{minipage}{\textwidth}
\begin{center}
\begin{tabular}{ll}
\toprule
Planet Name: & J\"{a}nni \\
Planet Type: & Gas Giant \\
Position in System: & 1 \\
Orbit: & 0.32 AU \\
Moons: & 3 \\
Diameter: & 130 000 km \\
Density: & 1.2 g/cm\textsuperscript{3} \\
Surface Gravity: & 2.2 G \\
Day Length: & 13 h \\
Year Length: & 0.20 terran years \\
Habitable: & No \\
\quad Atmospheric Density: & N/A \\
\quad Atmosphere Composition: & Hydrogen, Helium \\
\quad Surface Temperature: & N/A \\
\quad Percent Surface Water: & N/A \\
\quad Highest Life Form: & N/A \\

\bottomrule
\end{tabular}
\end{center}
\end{minipage}

\section{Kakkuri}

\newthought{Kakkuri} is the second planet of Kiiminki system.

\bigskip
\begin{minipage}{\textwidth}
\begin{center}
\begin{tabular}{ll}
\toprule
Planet Name: & Kakkuri \\
Planet Type: & Terrestrial \\
Position in System: & 2 \\
Orbit: & 0.56 AU \\
Moons: & 0 \\
Diameter: & 9 500 km \\
Density: & 4.18 g/cm\textsuperscript{3} \\
Surface Gravity: & 0.57 G \\
Day Length: & 18 h \\
Year Length: & 0.47 terran years \\
Habitable: & No \\
\quad Atmospheric Density: & Low \\
\quad Atmosphere Composition: & Nitrogen, Carbon Dioxide \\
\quad Surface Temperature: & 317K \\
\quad Percent Surface Water: & 0\% \\
\quad Highest Life Form: & N/A \\

\bottomrule
\end{tabular}
\end{center}
\end{minipage}

\section{Kuolimo}

\newthought{Kuolimo} is the third planet of Kiiminki system.

\bigskip
\begin{minipage}{\textwidth}
\begin{center}
\begin{tabular}{ll}
\toprule
Planet Name: & Kuolimo \\
Planet Type: & Terrestrial \\
Position in System: & 3 \\
Orbit: & 0.8 AU \\
Moons: & 0 \\
Diameter: & 12 500 km \\
Density: & 4.18 g/cm\textsuperscript{3} \\
Surface Gravity: & 0.74 G \\
Day Length: & 25 h \\
Year Length: & 0.80 terran years \\
Habitable: & Yes \\
\quad Atmospheric Density: & Normal \\
\quad Atmosphere Composition: & Nitrogen, Oxygen \\
\quad Surface Temperature: & 265K \\
\quad Percent Surface Water: & 70\% \\
\quad Highest Life Form: & Mammals \\

\bottomrule
\end{tabular}
\end{center}
\end{minipage}

\section{Jalanti}

\newthought{Jalanti} is the fourth planet of Kiiminki system.

\bigskip
\begin{minipage}{\textwidth}
\begin{center}
\begin{tabular}{ll}
\toprule
Planet Name: & Jalanti \\
Planet Type: & Terrestrial \\
Position in System: & 4 \\
Orbit: & 1.28 AU \\
Moons: & 2 \\
Diameter: & 11 500 km \\
Density: & 6.33 g/cm\textsuperscript{3} \\
Surface Gravity: & 1.04 G \\
Day Length: & 24 h \\
Year Length: & 1.62 terran years \\
Habitable: & No \\
\quad Atmospheric Density: & Low \\
\quad Atmosphere Composition: & Ammonia \\
\quad Surface Temperature: & 210K \\
\quad Percent Surface Water: & 0\% \\
\quad Highest Life Form: & N/A \\

\bottomrule
\end{tabular}
\end{center}
\end{minipage}

\section{Iso-Lankko}

\newthought{Iso-Lankko} is the fifth planet of Kiiminki system.

\bigskip
\begin{minipage}{\textwidth}
\begin{center}
\begin{tabular}{ll}
\toprule
Planet Name: & Iso-Lankko \\
Planet Type: & Ice Giant \\
Position in System: & 5 \\
Orbit: & 2.24 AU \\
Moons: & 9 \\
Diameter: & 45 000 km \\
Density: & 1.7 g/cm\textsuperscript{3} \\
Surface Gravity: & 1.09 G \\
Day Length: & 15 h \\
Year Length: & 3.76 terran years \\
Habitable: & No \\
\quad Atmospheric Density: & N/A \\
\quad Atmosphere Composition: & Methane, Ammonia \\
\quad Surface Temperature: & N/A \\
\quad Percent Surface Water: & N/A \\
\quad Highest Life Form: & N/A \\

\bottomrule
\end{tabular}
\end{center}
\end{minipage}


\section{Keyritty}

\newthought{Keyritty} is the sixth planet of Kiiminki system.

\bigskip
\begin{minipage}{\textwidth}
\begin{center}
\begin{tabular}{ll}
\toprule
Planet Name: & Keyritty \\
Planet Type: & Ice Giant \\
Position in System: & 6 \\
Orbit: & 4.16 AU \\
Moons: & 10 \\
Diameter: & 35 000 km \\
Density: & 1.9 g/cm\textsuperscript{3} \\
Surface Gravity: & 0.94 G \\
Day Length: & 13 h \\
Year Length: & 9.50 terran years \\
Habitable: & No \\
\quad Atmospheric Density: & N/A \\
\quad Atmosphere Composition: & Methane, Ammonia \\
\quad Surface Temperature: & N/A \\
\quad Percent Surface Water: & N/A \\
\quad Highest Life Form: & N/A \\

\bottomrule
\end{tabular}
\end{center}
\end{minipage}


\section{Meiko}

\newthought{Meiko} is the seventh planet of Kiiminki system.

\bigskip
\begin{minipage}{\textwidth}
\begin{center}
\begin{tabular}{ll}
\toprule
Planet Name: & Meiko \\
Planet Type: & Terrestrial \\
Position in System: & 7 \\
Orbit: & 8.0 AU \\
Moons: & 4 \\
Diameter: & 14 500 km \\
Density: & 4.78 g/cm\textsuperscript{3} \\
Surface Gravity: & 0.98 G \\
Day Length: & 22 h \\
Year Length: & 25.34 terran years \\
Habitable: & No \\
\quad Atmospheric Density: & Very High \\
\quad Atmosphere Composition: & Carbon Dioxide, Nitrogen \\
\quad Surface Temperature: & 537K \\
\quad Percent Surface Water: & 0\% \\
\quad Highest Life Form: & N/A \\

\bottomrule
\end{tabular}
\end{center}
\end{minipage}


\section{Maintainen}

\newthought{Maintainen} is the eigth planet of Kiiminki system.

\bigskip
\begin{minipage}{\textwidth}
\begin{center}
\begin{tabular}{ll}
\toprule
Planet Name: & Maintainen \\
Planet Type: & Gas Giant \\
Position in System: & 8 \\
Orbit: & 15.68 AU \\
Moons: & 16 \\
Diameter: & 70 000 km \\
Density: & 1.00 g/cm\textsuperscript{3} \\
Surface Gravity: & 1.00 G \\
Day Length: & 12 h \\
Year Length: & 69.55 terran years \\
Habitable: & No \\
\quad Atmospheric Density: & N/A \\
\quad Atmosphere Composition: & Hydrogen, Helium \\
\quad Surface Temperature: & N/A \\
\quad Percent Surface Water: & N/A \\
\quad Highest Life Form: & N/A \\

\bottomrule
\end{tabular}
\end{center}
\end{minipage}


\section{Kuorinka}

\newthought{Kuorinka} is the ninth planet of Kiiminki system.

\bigskip
\begin{minipage}{\textwidth}
\begin{center}
\begin{tabular}{ll}
\toprule
Planet Name: & Kuorinka \\
Planet Type: & Dwarf Terrestrial \\
Position in System: & 9 \\
Orbit: & 31.04 AU \\
Moons: & 0 \\
Diameter: & 1 800 km \\
Density: & 1.00 g/cm\textsuperscript{3} \\
Surface Gravity: & 0.03 G \\
Day Length: & 24 h \\
Year Length: & 193.70 terran years \\
Habitable: & No \\
\quad Atmospheric Density: & Vacuum \\
\quad Atmosphere Composition: & N/A \\
\quad Surface Temperature: & 42K \\
\quad Percent Surface Water: & 0\% \\
\quad Highest Life Form: & N/A \\

\bottomrule
\end{tabular}
\end{center}
\end{minipage}

\clearpage

\chapter{Fifty Words for Snow}

\newthought{People of Kiiminki system} had fifty words for snow and
J\o rdis hated each and every of them with burning passion. He hated
\emph{polanne}, solidified and partly frozen snow. He hated \emph{viti}, 
freshly snowed light snow. And especially he hated \emph{kinos}, snow 
that strong wind had pushed into dune shapes. In short, J\o rdis hated 
snow in general. The problem was, that in Kiiminki prime, snow was 
everywhere.

J\o rdis sighed and continued inspecting the Snow Moose, a huge
tracked transport that had been specifically designed for arctic
environment. If there was something that he hated more than snow, it
was having his vehicle to break down in the middle of nowhere, with
nothing else than snow around him.

Snow Moose was parked in a large hall with few other vehicles.
Floodlights provided light and he had turned vehicle's work lights on
too. Hall was almost warm, just a tad below freezing temperature.
There was no point letting snowy vehicles to warm so much that the
snow would melt and start rusting the steel. J\o rdis rubbed his hands
together briefly to warm them up and continued with the inspection.
Soon he would be done and hopefully Pettersson would be there to
join the ride to the outpost where they were to deliver some
supplies.

\newthought{Diesel engine} started to rumble as J\o rdis started the
engine and performed the final check up before unhooking the heating
cables and closing the engine hatch. Pettersson was late, but that was
nothing new. J\o rdis let the engine idle and listened to the sound.
Then he saw Pettersson finally enter from the side door and wave him
before starting to open the big double doors. Finally, they were about
to start their supply run to outpost 15.

J\o rdis turned on the headlights and slowly drove the Snow Moose out
into the darkness. It was early morning and the temperature was frisky
-25\celsius. J\o rdis dreamed of living on a sandy desert planet with
twin suns, as he waited for Pettersson to close the doors behind him 
and climb into the cabin. It was time to start the supply run.

\clearpage

\chapter{Support Vehicles}
\label{ch:support-vehicles}


\newthought{Support vehicles} form an essential part of any planet's
backbone. They might not be as flashy as machines of war, but without
them a planet would be in trouble.

This section will detail some new support vehicles that have been
designed specifically to work in the arctic environment.

\clearpage

\section{Snow Mover}
\newthought{Snow Mover} is a cheap and simple truck aimed for snow clearing and
removal. Two man crew is needed to effectively manage clearing equipment
and large searchlights mounted on the roof of the truck.

Snow Mover is the first dedicated snow clearing equipment by Drago ltd. As
such, it was based on very primitive technology as the goal was to build
cheap, sturdy and easy to maintain vehicle.

The vehicle didn't gain very widespread adoption. Partly because the very
specialized niche and partly because it was pretty terrible in clearing
snow. Wheeled design tended to get stuck in snow and it suffered from
other malfunctions too\marginnote{The engine was notoriously slow to accelerate 
and would use disproportionate amounts of gas when driven at full speed.
On top of that, Snow Mover lacks even the most basic armour, making it
vulnerable to accidents.}.

\bigskip
\begin{minipage}{\textwidth}
\begin{center}
\begin{tabular}{llll}
\toprule
Type: & Snow Mover & \\
Chassis Type: & Wheeled (Medium) & \\
Mass: & 10 tons & \\
Tech base: & Inner Sphere & \\
Equipment Rating: & C-C-C & \\
Availability: & X-B-B & \\
Production Year: & 3025 & \\
Cost: & 88 200? & \\
Battle Value: & & \\
Manufacturer: & Drago, Ltd & \\
Equipment & & Mass \\
\quad Chassis/Controls: & & 2.0 \\
\quad Engine/Trans: & ICE  & 3.0 \\
\quad Fuel: & 1660 km (Petrochemical) & 0.5 \\
\quad Speed: & \multicolumn{2}{l}{4/6} \\
Structure and Armour & & \\
\quad Armor Factor (N/A): & N/A & N/A \\
\quad & Internal & Armor \\
\quad Front & 1 & 0 \\
\quad R/L Side & 1 & 0 \\
\quad Rear & 1 & 0 \\
Armament and Capabilities: & & \\
\multicolumn{2}{l}{\quad Bulldozer, Front} & 2.0 \\
\multicolumn{2}{l}{\quad Dumper, backfacing} & 50kg \\
\multicolumn{2}{l}{\quad 2 Search lights, Front and Back} & 0.5 each \\

\multicolumn{3}{l}{Crew: 2} \\
Cargo: & & \\
\multicolumn{3}{l}{\quad 1 ton bulk (1 ton), with a dumper} \\

Notes: & & \\
\multicolumn{3}{l}{\quad Easy to Maintain, Gas Hog, Poor Performance} \\
\bottomrule
\end{tabular}
\end{center}
\end{minipage}

\clearpage
\section{Tracked Snow Mover}
\newthought{Not satisfied} with the original Snow Mover, Drago Ltd set to design an
improved version. The new version used tracks and was specifically
designed for snow operations. Using the easy to maintain diesel power
plant meant that the vehicle was slower than before, with a top speed
of 54 km/h\marginnote{Originally engineers wanted to have the vehicle to have
a top speed of 64.8 km/h, but that simply wasn't possible with the chosen
engine type.} and was burning through fuel at a high rate when driven at top
speed.

Tracked snow mover was fitted with half a ton of BAR 4 armour, that would
protect it against accidental glances and falling debris. While not
originally designed to work as a bulldozer, tracked snow mover could be used
in such an operation too.

\bigskip
\begin{minipage}{\textwidth}
\begin{center}
\begin{tabular}{llll}
\toprule
Type: & Tracked Snow Mover & \\
Chassis Type: & Tracked (Medium) & \\
Mass: & 25 tons & \\
Tech base: & Inner Sphere & \\
Equipment Rating: & C-C-C & \\
Availability: & X-B-B & \\
Production Year: & 3030 & \\
Cost: & 228 625 & \\
Battle Value: & & \\
Manufacturer: & Drago, Ltd & \\
Equipment & & Mass \\
\quad Chassis/Controls: & & 8.0 \\
\quad Engine/Trans: & ICE & 8.0 \\
\quad Fuel: & 625 km (Petrochemical) & 0.5 \\
\quad Speed: & \multicolumn{2}{l}{3/5} \\
Structure and Armour & & \\
\quad Armor Factor (BAR 4): & 31 & 1.0 \\
\quad & Internal & Armor \\
\quad Front & 3 & 10 \\
\quad R/L Side & 3 & 7 \\
\quad Rear & 3 & 7 \\

Armament and Capabilities: & & \\
\multicolumn{2}{l}{\quad Bulldozer, Front} & 2.0 \\
\multicolumn{2}{l}{\quad Dumper, backfacing} & 50kg \\
\multicolumn{2}{l}{\quad Search light, Front} & 0.5 \\

\multicolumn{3}{l}{Crew: 2} \\
Cargo: & & \\
\multicolumn{3}{l}{\quad 1 ton bulk (1 ton), with a dumper} \\

Notes: & & \\
\multicolumn{3}{l}{\quad Snowmobile, Easy to Maintain, Gas Hog} \\

\bottomrule
\end{tabular}
\end{center}
\end{minipage}



\chapter{Combat Vehicles}
\label{ch:combat-vehicles}


\newthought{Combat vehicles} are the bread and butter of any army.

This section will detail some new combat vehicles that have been
designed specifically to work in the arctic environment.

\clearpage



\chapter{IndustrialMechs}
\label{ch:industrialmechs}


\newthought{IndustrialMechs} are much more versatile than support vehicles
when it comes to traveling on a difficult terrain. They are also quite
a bit more expensive and harder to maintain.

This section will detail some new IndustrialMechs that have been
designed specifically to work in the arctic environment.

\clearpage
\section{Digger DGR-1A}
\newthought{Digger} was designed as a mining 'mech, with specific focus
on breaking stone and loading it onto other 'mech or vehicle. It does not
have any cargo capacity, but sports rock cutter and backhoe.

\marginnote[+7cm]{While fusion engine has cheap running costs and long range,
maintenance is a lot more difficult than more conventional engines require.
Not all customers were happy about the decision to use it.}
\marginnote[+1.8cm]{DGR-1A uses standard armour, which protects it well against
rigours of mining work.}
\bigskip
\begin{minipage}{\textwidth}
\begin{center}
\begin{tabular}{llll}
\toprule
Type: & Digger DGR-1A & \\
Chassis Type: & Biped IndustrialMech & \\
Mass: & 25 tons & \\
Tech base: & Inner Sphere & \\
Rating \& Availability: & D/D-D-A & \\
Production Year: & 3025 & \\
Cost: & 984 052 & \\
Battle Value: & 213 & \\
Manufacturer: & ? & \\
Equipment & & Mass \\
\quad Chassis/Controls: & Industrial & 5.0 \\
\quad Armor Factor: & Standard & 2.5 \\
\quad Engine/Trans: & Fusion Engine & 3.0 \\
\quad Fuel: & ? & ? \\
\quad Speed: & \multicolumn{2}{l}{4/6/0} \\
\quad Heatsinks: & Single Heat Sink & 0.0 \\
\quad Heatsink locations: & 2 CT, 2 LL, 2 RL & \\
\quad Gyro: & Standard & 1.0 \\
\quad Cockpit: & Industrial & 3.0 \\
\quad Actuators: & Left: SH+UA+LA & Right: SH+UA+LA \\
Structure and Armour & Internal & Armor \\
\quad Head & 3 & 7 \\
\quad Center Torso & 8 & 6 \\
\quad Center Torso (rear) & - & 1 \\
\quad L/R Torso & 6 & 5 \\
\quad L/R Torso (rear) & - & 1 \\
\quad L/R Arm & 4 & 3 \\
\quad L/R Leg & 6 & 4 \\

Armament and Capabilities: & & \\
\quad Searchlight & Head & 0.5 \\
\quad Rock cutter & Right Arm & 5.0 \\
\quad Backhoe & Left Arm & 5.0 \\

Notes: & & \\
\multicolumn{3}{l}{\quad } \\

\bottomrule
\end{tabular}
\end{center}
\end{minipage}

\clearpage
\section{Digger DGR-1D}
\newthought{Digger 1D} is a I.C.E. version of the earlier A-model. 
I.C.E. and commercial armour are quite a bit cheaper and there
is no too big drop in the performance. As a result D-model is 
very popular one, especially in the operations that have very limited 
resource pool.

\marginnote[+7cm]{Performance of I.C.E. engine may not be as good as
fusion engine's, but the cost is signigically lower. The problem is that
fuel consumption is far more greater.}
\marginnote[+2cm]{Commercial armour doesn't protect as well as the
standard one, but it is usually more than enough for a mining 'mech.}
\bigskip
\begin{minipage}{\textwidth}
\begin{center}
\begin{tabular}{llll}
\toprule
Type: & Digger DGR-1D & \\
Chassis Type: & Biped IndustrialMech & \\
Mass: & 25 tons & \\
Tech base: & Inner Sphere & \\
Rating \& Availability: & D/D-D-A & \\
Production Year: & 3025 & \\
Cost: & 818 391 & \\
Battle Value: & 143 & \\
Manufacturer: & ? & \\
Equipment & & Mass \\
\quad Chassis/Controls: & Industrial & 5.0 \\
\quad Armor Factor: & Commercial (BAR 5) & 1.5 \\
\quad Engine/Trans: & I.C.E. & 4.0 \\
\quad Fuel: & 600 km (Petrochemical) & 0.0 \\
\quad Speed: & \multicolumn{2}{l}{3/5/0} \\
\quad Heatsinks: & Single Heat Sink & 0.0 \\
\quad Heatsink locations: &  & \\
\quad Gyro: & Standard & 1.0 \\
\quad Cockpit: & Industrial & 3.0 \\
\quad Actuators: & Left: SH+UA+LA & Right: SH+UA+LA \\
Structure and Armour & Internal & Armor \\
\quad Head & 3 & 6 \\
\quad Center Torso & 8 & 5 \\
\quad Center Torso (rear) & - & 1 \\
\quad L/R Torso & 6 & 4 \\
\quad L/R Torso (rear) & - & 1 \\
\quad L/R Arm & 4 & 3 \\
\quad L/R Leg & 6 & 4 \\

Armament and Capabilities: & & \\
\quad Searchlight & Head & 0.5 \\
\quad Rock cutter & Right Arm & 5.0 \\
\quad Backhoe & Left Arm & 5.0 \\

Notes: & & \\
\multicolumn{3}{l}{\quad } \\

\bottomrule
\end{tabular}
\end{center}
\end{minipage}


\clearpage
\section{Digger DGR-1F}
\newthought{Digger 1F} is a fuel-cell variant of the earlier A-model.
Performance is on par with the DGR-1A in all other areas
execpt on operation range and armour. Fuel-cell model was never really
popular since operations with cash to spare generally selected fusion
engine one, while others opted for I.C.E.

\marginnote[+7cm]{Fuel-Cell engine is only slightly heavier than
the fusion one and the performance is pretty much equal.}

\bigskip
\begin{minipage}{\textwidth}
\begin{center}
\begin{tabular}{llll}
\toprule
Type: & Digger DGR-1F & \\
Chassis Type: & Biped IndustrialMech & \\
Mass: & 25 tons & \\
Tech base: & Inner Sphere & \\
Rating \& Availability: & D/D-D-A & \\
Production Year: & 3025 & \\
Cost: & 911 271 & \\
Battle Value: & 147 & \\
Manufacturer: & ? & \\
Equipment & & Mass \\
\quad Chassis/Controls: & Industrial & 5.0 \\
\quad Armor Factor: & Commercial (BAR 5) & 1.5 \\
\quad Engine/Trans: & Fuel-Cell & 4.0 \\
\quad Fuel: & 450 km (Chemical) & 0.0 \\
\quad Speed: & \multicolumn{2}{l}{4/6/0} \\
\quad Heatsinks: & Single Heat Sink & 0.0 \\
\quad Heatsink locations: &  & \\
\quad Gyro: & Standard & 1.0 \\
\quad Cockpit: & Industrial & 3.0 \\
\quad Actuators: & Left: SH+UA+LA & Right: SH+UA+LA \\
Structure and Armour & Internal & Armor \\
\quad Head & 3 & 6 \\
\quad Center Torso & 8 & 5 \\
\quad Center Torso (rear) & - & 1 \\
\quad L/R Torso & 6 & 4 \\
\quad L/R Torso (rear) & - & 1 \\
\quad L/R Arm & 4 & 3 \\
\quad L/R Leg & 6 & 4 \\

Armament and Capabilities: & & \\
\quad Searchlight & Head & 0.5 \\
\quad Rock cutter & Right Arm & 5.0 \\
\quad Backhoe & Left Arm & 5.0 \\

Notes: & & \\
\multicolumn{3}{l}{\quad } \\

\bottomrule
\end{tabular}
\end{center}
\end{minipage}

\chapter{BattleMechs}
\label{ch:battlemechs}


\newthought{BattleMechs} are kings of the battle ground. 

This section will detail some new BattleMechs that have been
designed specifically to work in the arctic environment.

\clearpage

%%
% The back matter contains appendices, bibliographies, indices, glossaries, etc.







\backmatter

% \bibliography{sample-handout}
% \bibliographystyle{plainnat}


\printindex

\end{document}


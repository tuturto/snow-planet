\documentclass{tufte-book}

\hypersetup{colorlinks}% uncomment this line if you prefer colored hyperlinks (e.g., for onscreen viewing)

%%
% Book metadata
\title{TRO: Snow Planet}
\author{Tuukka Turto}
\publisher{N/A}

%%
% If they're installed, use Bergamo and Chantilly from www.fontsite.com.
% They're clones of Bembo and Gill Sans, respectively.
%\IfFileExists{bergamo.sty}{\usepackage[osf]{bergamo}}{}% Bembo
%\IfFileExists{chantill.sty}{\usepackage{chantill}}{}% Gill Sans

%\usepackage{microtype}

%%
% Just some sample text
\usepackage{lipsum}
\usepackage{geometry}

\geometry{paper=a4paper}

%%
% For nicely typeset tabular material
\usepackage{booktabs}

%%
% For graphics / images
\usepackage{graphicx}
\setkeys{Gin}{width=\linewidth,totalheight=\textheight,keepaspectratio}
\graphicspath{{graphics/}}

% The fancyvrb package lets us customize the formatting of verbatim
% environments.  We use a slightly smaller font.
\usepackage{fancyvrb}
\fvset{fontsize=\normalsize}

%%
% Prints argument within hanging parentheses (i.e., parentheses that take
% up no horizontal space).  Useful in tabular environments.
\newcommand{\hangp}[1]{\makebox[0pt][r]{(}#1\makebox[0pt][l]{)}}

%%
% Prints an asterisk that takes up no horizontal space.
% Useful in tabular environments.
\newcommand{\hangstar}{\makebox[0pt][l]{*}}

%%
% Prints a trailing space in a smart way.
\usepackage{xspace}

% Prints the month name (e.g., January) and the year (e.g., 2008)
\newcommand{\monthyear}{%
  \ifcase\month\or January\or February\or March\or April\or May\or June\or
  July\or August\or September\or October\or November\or
  December\fi\space\number\year
}


% Prints an epigraph and speaker in sans serif, all-caps type.
\newcommand{\openepigraph}[2]{%
  %\sffamily\fontsize{14}{16}\selectfont
  \begin{fullwidth}
  \sffamily\large
  \begin{doublespace}
  \noindent\allcaps{#1}\\% epigraph
  \noindent\allcaps{#2}% author
  \end{doublespace}
  \end{fullwidth}
}

% Inserts a blank page
\newcommand{\blankpage}{\newpage\hbox{}\thispagestyle{empty}\newpage}

\usepackage{units}

% Typesets the font size, leading, and measure in the form of 10/12x26 pc.
\newcommand{\measure}[3]{#1/#2$\times$\unit[#3]{pc}}

% Macros for typesetting the documentation
\newcommand{\hlred}[1]{\textcolor{Maroon}{#1}}% prints in red
\newcommand{\hangleft}[1]{\makebox[0pt][r]{#1}}
\newcommand{\hairsp}{\hspace{1pt}}% hair space
\newcommand{\hquad}{\hskip0.5em\relax}% half quad space
\newcommand{\TODO}{\textcolor{red}{\bf TODO!}\xspace}
\newcommand{\ie}{\textit{i.\hairsp{}e.}\xspace}
\newcommand{\eg}{\textit{e.\hairsp{}g.}\xspace}
\newcommand{\na}{\quad--}% used in tables for N/A cells
\providecommand{\XeLaTeX}{X\lower.5ex\hbox{\kern-0.15em\reflectbox{E}}\kern-0.1em\LaTeX}
\newcommand{\tXeLaTeX}{\XeLaTeX\index{XeLaTeX@\protect\XeLaTeX}}
% \index{\texttt{\textbackslash xyz}@\hangleft{\texttt{\textbackslash}}\texttt{xyz}}
\newcommand{\tuftebs}{\symbol{'134}}% a backslash in tt type in OT1/T1
\newcommand{\doccmdnoindex}[2][]{\texttt{\tuftebs#2}}% command name -- adds backslash automatically (and doesn't add cmd to the index)
\newcommand{\doccmddef}[2][]{%
  \hlred{\texttt{\tuftebs#2}}\label{cmd:#2}%
  \ifthenelse{\isempty{#1}}%
    {% add the command to the index
      \index{#2 command@\protect\hangleft{\texttt{\tuftebs}}\texttt{#2}}% command name
    }%
    {% add the command and package to the index
      \index{#2 command@\protect\hangleft{\texttt{\tuftebs}}\texttt{#2} (\texttt{#1} package)}% command name
      \index{#1 package@\texttt{#1} package}\index{packages!#1@\texttt{#1}}% package name
    }%
}% command name -- adds backslash automatically
\newcommand{\doccmd}[2][]{%
  \texttt{\tuftebs#2}%
  \ifthenelse{\isempty{#1}}%
    {% add the command to the index
      \index{#2 command@\protect\hangleft{\texttt{\tuftebs}}\texttt{#2}}% command name
    }%
    {% add the command and package to the index
      \index{#2 command@\protect\hangleft{\texttt{\tuftebs}}\texttt{#2} (\texttt{#1} package)}% command name
      \index{#1 package@\texttt{#1} package}\index{packages!#1@\texttt{#1}}% package name
    }%
}% command name -- adds backslash automatically
\newcommand{\docopt}[1]{\ensuremath{\langle}\textrm{\textit{#1}}\ensuremath{\rangle}}% optional command argument
\newcommand{\docarg}[1]{\textrm{\textit{#1}}}% (required) command argument
\newenvironment{docspec}{\begin{quotation}\ttfamily\parskip0pt\parindent0pt\ignorespaces}{\end{quotation}}% command specification environment
\newcommand{\docenv}[1]{\texttt{#1}\index{#1 environment@\texttt{#1} environment}\index{environments!#1@\texttt{#1}}}% environment name
\newcommand{\docenvdef}[1]{\hlred{\texttt{#1}}\label{env:#1}\index{#1 environment@\texttt{#1} environment}\index{environments!#1@\texttt{#1}}}% environment name
\newcommand{\docpkg}[1]{\texttt{#1}\index{#1 package@\texttt{#1} package}\index{packages!#1@\texttt{#1}}}% package name
\newcommand{\doccls}[1]{\texttt{#1}}% document class name
\newcommand{\docclsopt}[1]{\texttt{#1}\index{#1 class option@\texttt{#1} class option}\index{class options!#1@\texttt{#1}}}% document class option name
\newcommand{\docclsoptdef}[1]{\hlred{\texttt{#1}}\label{clsopt:#1}\index{#1 class option@\texttt{#1} class option}\index{class options!#1@\texttt{#1}}}% document class option name defined
\newcommand{\docmsg}[2]{\bigskip\begin{fullwidth}\noindent\ttfamily#1\end{fullwidth}\medskip\par\noindent#2}
\newcommand{\docfilehook}[2]{\texttt{#1}\index{file hooks!#2}\index{#1@\texttt{#1}}}
\newcommand{\doccounter}[1]{\texttt{#1}\index{#1 counter@\texttt{#1} counter}}

% Generates the index
\usepackage{makeidx}
\makeindex

\begin{document}

% Front matter
\frontmatter

% r.3 full title page
\maketitle


% v.4 copyright page
\newpage
\begin{fullwidth}
~\vfill
\thispagestyle{empty}
\setlength{\parindent}{0pt}
\setlength{\parskip}{\baselineskip}
Copyright \copyright\ \the\year\ \thanklessauthor

\par\smallcaps{Published by \thanklesspublisher}

\par\smallcaps{tufte-latex.googlecode.com}

\par Licensed under CC BY-SA 4.0.\index{license}

\par\textit{First printing, \monthyear}
\end{fullwidth}

% r.5 contents
\setcounter{tocdepth}{2}
\tableofcontents

%%
% Start the main matter (normal chapters)
\mainmatter


\chapter{Support Vehicles}
\label{ch:support-vehicles}


\newthought{Support vehicles} form an essential part of any planet's
backbone. They might not be as flashy as machines of war, but without
them a planet would be in trouble.

This section will detail some new support vehicles that have been
designed specifically to work in the arctic environment.

\clearpage

\section{Snow Mover}
\newthought{Snow Mover} is a cheap and simple truck aimed for snow clearing and
removal. Two man crew is needed to effectively manage clearing equipment
and large searchlights mounted on the roof of the truck.

Snow Mover is the first dedicated snow clearing equipment by Drago ltd. As
such, it was based on very primitive technology as the goal was to build
cheap, sturdy and easy to maintain vehicle.

The vehicle didn't gain very widespread adoption. Partly because the very
specialized niche and partly because it was pretty terrible in clearing
snow. Wheeled design tended to get stuck in snow and it suffered from
other malfunctions too\footnote{The engine was notoriously slow to accelerate 
and would use disproportionate amounts of gas when driven at full speed.}.

\bigskip
\begin{minipage}{\textwidth}
\begin{center}
\begin{tabular}{llll}
\toprule
Type: & Snow Mover & \\
Chassis Type: & Wheeled (Medium) & \\
Mass: & 10 tons & \\
Tech base: & Inner Sphere & \\
Equipment Rating: & C-C-C & \\
Availability: & X-B-B & \\
Production Year: & 3025 & \\
Cost: & 88 200? & \\
Battle Value: & & \\
Manufacturer: & Drago, Ltd & \\
Equipment & & Mass \\
\quad Chassis/Controls: & & 2.0 \\
\quad Engine/Trans: & ICE  & 3.0 \\
\quad Fuel: & 1660 km (Petrochemical) & 0.5 \\
\quad Speed: & \multicolumn{2}{l}{4/6} \\
Structure and Armour & & \\
\quad Armor Factor (N/A): & N/A & N/A \\
\quad & Internal & Armor \\
\quad Front & 1 & 0 \\
\quad R/L Side & 1 & 0 \\
\quad Rear & 1 & 0 \\
Armament and Capabilities: & & \\
\multicolumn{2}{l}{\quad Bulldozer, Front} & 2.0 \\
\multicolumn{2}{l}{\quad Dumper, backfacing} & 50kg \\
\multicolumn{2}{l}{\quad 2 Search lights, Front and Back} & 0.5 each \\

\multicolumn{3}{l}{Crew: 2} \\
Cargo: & & \\
\multicolumn{3}{l}{\quad 1 ton bulk (1 ton), with a dumper} \\

Notes: & & \\
\multicolumn{3}{l}{\quad Easy to Maintain, Gas Hog, Poor Performance} \\
\bottomrule
\end{tabular}
\end{center}
\end{minipage}

\clearpage
\section{Tracked Snow Mover}
\newthought{Not satisfied} with the original Snow Mover, Drago Ltd set to design an
improved version. The new version used tracks and was specifically
designed for snow operations. Using the easy to maintain diesel power
plant meant that the vehicle was slower than before, with a top speed
of 54 km/h\footnote{Originally engineers wanted to have the vehicle to have
a top speed of 64.8 km/h, but that simply wasn't possible with the chosen
engine type.} and was burning through fuel at a high rate when driven at top
speed.

Tracked snow mover was fitted with half a ton of BAR 4 armour, that would
protect it against accidental glances and falling debris. While not
originally designed to work as a bulldozer, tracked snow mover could be used
in such an operation too.

\bigskip
\begin{minipage}{\textwidth}
\begin{center}
\begin{tabular}{llll}
\toprule
Type: & Tracked Snow Mover & \\
Chassis Type: & Tracked (Medium) & \\
Mass: & 25 tons & \\
Tech base: & Inner Sphere & \\
Equipment Rating: & C-C-C & \\
Availability: & X-B-B & \\
Production Year: & 3030 & \\
Cost: & 228 625 & \\
Battle Value: & & \\
Manufacturer: & Drago, Ltd & \\
Equipment & & Mass \\
\quad Chassis/Controls: & & 8.0 \\
\quad Engine/Trans: & ICE & 8.0 \\
\quad Fuel: & 625 km (Petrochemical) & 0.5 \\
\quad Speed: & \multicolumn{2}{l}{3/5} \\
Structure and Armour & & \\
\quad Armor Factor (BAR 4): & 31 & 1.0 \\
\quad & Internal & Armor \\
\quad Front & 3 & 10 \\
\quad R/L Side & 3 & 7 \\
\quad Rear & 3 & 7 \\

Armament and Capabilities: & & \\
\multicolumn{2}{l}{\quad Bulldozer, Front} & 2.0 \\
\multicolumn{2}{l}{\quad Dumper, backfacing} & 50kg \\
\multicolumn{2}{l}{\quad Search light, Front} & 0.5 \\

\multicolumn{3}{l}{Crew: 2} \\
Cargo: & & \\
\multicolumn{3}{l}{\quad 1 ton bulk (1 ton), with a dumper} \\

Notes: & & \\
\multicolumn{3}{l}{\quad Snowmobile, Easy to Maintain, Gas Hog} \\

\bottomrule
\end{tabular}
\end{center}
\end{minipage}

% 

%%
% The back matter contains appendices, bibliographies, indices, glossaries, etc.







\backmatter

% \bibliography{sample-handout}
% \bibliographystyle{plainnat}


\printindex

\end{document}

